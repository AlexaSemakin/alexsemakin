\documentclass[10pt,a4paper,ragged2e,withhyper]{altacv}

\geometry{left=1.25cm,right=1.25cm,top=1.5cm,bottom=1.5cm,columnsep=1.2cm}

\usepackage{paracol}
\usepackage{qrcode}
\ifxetexorluatex
  \setmainfont{Roboto Slab}
  \setsansfont{Lato}
  \renewcommand{\familydefault}{\sfdefault}
\else
  \usepackage[english,russian]{babel}
  \usepackage[rm]{roboto}
  \usepackage[defaultsans]{lato}
  \usepackage{sourcesanspro}
  \renewcommand{\familydefault}{\sfdefault}
\fi

% Change the colours if you want to
\definecolor{SlateGrey}{HTML}{2E2E2E}
\definecolor{LightGrey}{HTML}{222222}
\definecolor{DarkPastelRed}{HTML}{450808}
\definecolor{PastelRed}{HTML}{8F0D0D}
\definecolor{GoldenEarth}{HTML}{E7D192}
\colorlet{name}{black}
\colorlet{tagline}{PastelRed}
\colorlet{heading}{DarkPastelRed}
\colorlet{headingrule}{GoldenEarth}
\colorlet{subheading}{PastelRed}
\colorlet{accent}{PastelRed}
\colorlet{emphasis}{SlateGrey}
\colorlet{body}{LightGrey}

% Change some fonts, if necessary
\renewcommand{\namefont}{\Huge\rmfamily\bfseries}
\renewcommand{\personalinfofont}{\footnotesize}
\renewcommand{\cvsectionfont}{\LARGE\rmfamily\bfseries}
\renewcommand{\cvsubsectionfont}{\large\bfseries}


% Change the bullets for itemize and rating marker
% for \cvskill if you want to
\renewcommand{\itemmarker}{{\small\textbullet}}
\renewcommand{\ratingmarker}{\faCircle}

%% Use (and optionally edit if necessary) this .tex if you
%% want to use an author-year reference style like APA(6)
%% for your publication list
% \input{pubs-authoryear}

%% Use (and optionally edit if necessary) this .tex if you
%% want an originally numerical reference style like IEEE
%% for your publication list
\input{pubs-num}

%% sample.bib contains your publications
\addbibresource{sample.bib}

\begin{document}
\name{Семакин Александр\\ Александрович}
\tagline{Предприниматель, руководитель проектов, бизнес/системный аналитик, разработчик ПО, тренер по спортивному программированию}
%% You can add multiple photos on the left or right
\photoR{2.8cm}{ava_new.jpg}
% \photoL{2.5cm}{Yacht_High,Suitcase_High}

\personalinfo{%
  \email{alexasemakin@gmail.com}
  \phone{+7(912)013-39-97}
  \location{Томск, РФ}
  \github{AlexaSemakin}
}

\makecvheader

\columnratio{0.65}

\begin{paracol}{2}
\cvsection{Опыт работы}

%\cvevent{Старший методист, преподаватель}{ООО ``Онлайн школа программирования''}5{Сентябрь 2020 -- июнь 2021}{Удалённо}

%\textcolor{PastelRed}{Обязанности в качестве старшего методиста по спортивному программированию:}
%\begin{itemize}
%\item Создание и актуализация методических материалов
%\item Обучение педагогов
%\item Проведение собеседования на должность педагога по спортивному программированию
%\item Формирование ответов на вопросы педагогов
%\item Проведение конференций по актуализации методических материалов, мастер-классов
%\end{itemize}
%\textcolor{PastelRed}{Обязанности в качестве преподавателя:}
%\begin{itemize}
%\item Проведение урока по методическому материалу
%\item Формирование отчета по каждому проведенному уроку
%\item Отправка обратной связи родителям по успеваемости учеников
%\item Проведение тестов, проверяющих знания учащегося
%\end{itemize}

%\divider

\cvevent{Руководитель онлайн школы, руководитель команды разработчиков, разработчик, преподаватель}{ИП ``Семакин Александр Александрович''}{Май 2021 -- Декабрь 2023}{Удалённо}

\textcolor{PastelRed}{Обязанности в качестве руководителя онлайн школы:}
\begin{itemize}
    \item Управление образовательным процессом школы с численностью 50+ учеников
    \item Разработка и актуализация учебных программ 
    \item Контроль качества обучения и мониторинг успеваемости учащихся
    \item Организация финансовых расчетов с преподавательским составом
    \item Развитие клиентской базы и проведение презентаций образовательных продуктов
    \item Формирование преподавательского состава: поиск, отбор и оценка кандидатов
    \item Проведение технических собеседований с потенциальными преподавателями 
    \item Онбординг новых преподавателей
\end{itemize}

\textcolor{PastelRed}{Обязанности в качестве руководителя команды разработчиков:}
\begin{itemize}
    \item Управление командой из 5 разработчиков C#/.NET
    \item Проектирование архитектуры приложений на базе .NET Core
    \item Разработка технических заданий и декомпозиция проектных задач
    \item Внедрение Agile-методологий и организация спринтов
    \item Проведение код-ревью и обеспечение качества кодовой базы
    \item Организация процесса непрерывной интеграции (CI/CD)
    \item Взаимодействие с заказчиками на всех этапах разработки
    \item Управление проектными рисками и сроками выполнения работ
\end{itemize}

\textcolor{PastelRed}{Обязанности в качестве преподавателя:}
\begin{itemize}
    \item Проведение индивидуальных и групповых занятий по программированию на C#
    \item Разработка учебных материалов и практических заданий
    \item Подготовка учеников к участию в профильных олимпиадах и конкурсах
    \item Проведение регулярной оценки знаний и навыков учащихся
    \item Предоставление развернутой обратной связи родителям
    \item Сопровождение учебных проектов учеников на C#
\end{itemize}

\textcolor{PastelRed}{Обязанности в качестве разработчика:}
\begin{itemize}
    \item Разработка веб-приложений на платформе ASP.NET Core
    \item Создание и поддержка микросервисной архитектуры
    \item Работа с базами данных MS SQL Server и Entity Framework
    \item Внедрение паттернов проектирования и SOLID принципов
    \item Ведение технической документации в Confluence
    \item Использование систем контроля версий Git и Azure DevOps
    \item Участие в code review и рефакторинге существующего кода
\end{itemize}

\medskip
\switchcolumn
\cvsection{О себе}
Дата рождения - 27.09.2000\\
Место рождения - г. Глазов, УР

\cvsection{Навыки}

Технические знания \linebreak\\
\cvtag{С\# (WF, WPF, Xamarin, Selenium)}\\
\cvtag{SQL}\\
\cvtag{Python}
\cvtag{SQL}
\cvtag{C}
\cvtag{C++ (QT Create)}\\
\cvtag{CSS}
\cvtag{JS}
\cvtag{HTML}\\
\cvtag{LaTeX}\cvtag{MarkDown} 
\cvtag{BPMN}\\
\cvtag{Linux (Ubuntu, Mac OS)}
\cvtag{Arduino}

\divider\smallskip

Увлечения \linebreak\\
\cvtag{Компьютерные игры} \\
\cvtag{Радиоэлектроника} \\
\cvtag{Столярное дело} \\
\cvtag{Робототехника} \\
\cvtag{Математика} \\
\cvtag{Шахматы} \\\

\divider\smallskip

Другое \linebreak\\
\cvtag{Быстрая обучаемость}\\
\cvtag{Исполнительность}\\
\cvtag{Коммуникабельность}


\cvsection{Языки}

\cvskill{Русский}{5}
\divider

\cvskill{Английский}{3}

\medskip

\cvsection{Образование}

\cvevent{Студент, 3 курс, бакалавриат}{ТГУ, ИПМКН}{сентябрь. 2019 -- н.в.}{}
Факультет прикладной математики и информатики,\\
Кафедра компьютерной безопасности
\end{paracol}

\cvevent{(Middle) Бизнес-системный аналитик}{ООО ``Умное пространство''}{Июнь 2023 -- Июнь 2024}{Удалённо}
\textcolor{PastelRed}{Обязанности:}
\begin{itemize}

    \item Коммуникации со стейкхолдерами, включая взаимодействие со смежными командами;
    \item Постоянное взаимодействие с продакт-менеджером и работа над развитием продукта;
    \item Разработка и документирование требований, их согласование и управление изменениями;
    \item Оптимизация и автоматизация кросс-функциональных процессов, моделирование бизнес-процессов;
    \item Концептуальное моделирование предметной области;
    \item Участие в проработке интеграционных взаимодействий (внутренних и внешних);
    \item Декомпозиция и постановка задач команде, а также подготовка дополнительных материалов, например, для проверки гипотез, продуктовых исследований, дизайна, аналитики данных и т.д.;
    \item Консультирование в процессе разработки и тестирования, приемка, демонстрация и новых продуктовых фич.
    \item Описания бизнес-процессов (AS IS, TO BE)
    \item Описание API методов на YAML (Swagger)

\end{itemize}

\divider

\cvevent{Генеральный директор}{ООО ``Латекстид''}{Ноябрь 2023 -- н.в.}{г. Томск, ул. Академическая 8/8}
\textcolor{PastelRed}{Обязанности:}
\begin{itemize}
    \item Разработка и реализация стратегии развития компании в сфере ИТ, определение ключевых направлений роста
    \item Построение эффективной организационной структуры и оптимизация бизнес-процессов
    \item Формирование и контроль исполнения бюджета компании, управление инвестициями в ИТ-инфраструктуру
    \item Руководство портфелем ИТ-проектов, внедрение инновационных технологических решений
    \item Развитие партнерских отношений с ключевыми клиентами и вендорами
    \item Обеспечение информационной безопасности и непрерывности бизнес-процессов
    \item Проведение переговоров с инвесторами и привлечение финансирования
    \item Представление интересов компании во внешних коммуникациях
    \item Анализ рынка и конкурентной среды, определение возможностей для развития бизнеса
\end{itemize}

\divider

\cvevent{Руководитель проектов}{ООО ``Иптимайзер''}{Июль 2024 -- н.в.}{г. Томск, ул. Яковлева 15}
\textcolor{PastelRed}{Обязанности:}
\begin{itemize}
    \item Разработка и написание кода на C\# для ключевых компонентов проекта
    \item Проведение код-ревью и обеспечение высокого качества кодовой базы
    \item Выполнение бизнес-анализа для определения требований и возможностей проекта
    \item Создание и поддержание актуальности документации по бизнес-процессам и требованиям
    \item Управление backlog'ом продукта и определение приоритетов разработки как Product Owner
    \item Формирование видения продукта и разработка стратегии его развития
    \item Взаимодействие с заинтересованными сторонами для сбора и уточнения требований
    \item Руководство командой разработчиков, включая распределение задач и контроль их выполнения
    \item Проведение регулярных встреч команды (daily stand-ups, sprint planning, retrospectives)
    \item Управление рисками проекта и решение возникающих проблем
    \item Обеспечение соблюдения сроков и бюджета проекта
    \item Мотивация и развитие членов команды
    \item Участие в архитектурных решениях и выборе технологий для проекта
    \item Обеспечение соответствия разрабатываемого продукта бизнес-целям компании
    \item Проведение демонстраций продукта для заказчиков и сбор обратной связи
\end{itemize}

\cvsection{Научные статьи}
\begin{enumerate}
    \item[1.] Шудегова В. К., Семакин А. А., Бронер В. И. \textcolor{PastelRed}{Анализ существующих систем дистанционного обучения для частных онлайн школ}: Информационные технологии и математическое моделирование (ИТММ- 2022): Материалы XXI Международной конференции имени А. Ф. Терпугова (25-29 октября 2022 г.). г. Томск: Издательство Томского государственного университета, 2023. С. 411-416

    \item[2.] Семакин А. А., Шудегова В. К., Бобров В. А. \textcolor{PastelRed}{О проблеме популяризации результатов исторических исследований}:  Информационные технологии и математическое моделирование (ИТММ- 2022): Материалы XXI Международной конференции имени А. Ф. Терпугова (25-29 октября 2022 г.). г. Томск: Издательство Томского государственного университета, 2023. С. 417-425
\end{enumerate}

\cvsection{Проекты}

\cvevent{Макрос Excel для учёта трат компании (завершено)}{ПО для оптимизации бизнес-процессов}{Заказчик: Рекламное агентство ``МаксимуМ'' г. Глазов}{}
Макрос позволяет упростить добавление новых клиентов в таблицу, быстро получать аналитику и высчитывать необходимые коэффициенты

\divider

\cvevent{Картотека сотрудников (завершено)}{ПО для оптимизации бизнес-процессов}{В интересах: АО ``Чепецкий механический завод''}{}
Программное обеспечение, необходимое для отдела продаж в целях учёта сотрудников Чепецкого механического завода. Функционал приложения:
\begin{itemize}
\item Сортировка всех сотрудников по полям
\item Поиск сотрудника в базе
\item Импорт и экспорт данных из базы в форматах word и excel
\item Создание аккаунтов с ограниченными правами доступа к данным и функционалу
\item Отслеживание действий каждого аккаунта в системе
\end{itemize}

\divider

\cvevent{Лендинг продукта (завершено)}{Продающий сайт}{Заказчик: ООО ``Альфа-Партнер''}{}
Одностраничный сайт с краткой информацией об услугах компании, блоком отзывов, ценообразованием (онлайн калькулятор цены). Также на сайте присутствует возможность оставить свои контакты для дальнейшей связи менеджера с клиентом

\divider

\cvevent{Информационное приложение (завершено)}{ПО для сенсорной панели}{В интересах: ПАО ``Газпром''}{}
Приложение - интерфейс для вывода медиаконтента различных типов таких как:
\begin{itemize}
\item Десктопные приложения (*.exe, *.msi и т.п.)
\item PDF файлы
\item WEB сайты
\item Видеоматериалы
\end{itemize}

\divider

\cvevent{ПО для удалённого управления ПК в классе (завершено)}{ПО для удалённого изменения конфигураций ПК}{В интересах: МБОУ ``ФМЛ''}{}
Программное обеспечение позволяет изменять конфигурации компьютера и ОС удалённо, такие как: 
\begin{itemize}
\item Запрет на изменение параметров персонализации (и каждой отдельной настройки)
\item Запрет на запуск приложений, находящихся в чёрном списке (или альтернативный вариант - запрещены все приложения, кроме находящихся в белом списке)
\item Запрет на изменение файлов и структуры директории (либо во всей системе, либо в части каталогов)
\item Создание резервной копии директории
\item Восстановление каталога по резервной копии
\end{itemize}

\divider

\cvevent{Чат-бот для проведения олимпиад (завершено)}{ПО для проведения дистанционных соревнований}{В партнёрстве с НИ ``ТГУ''}{}
Чат-бот, позволяющий проводить дистанционные олимпиады по разным дисциплинам и от разных организаторов. Функционал бота позволяет проводить олимпиады, в которых участники фотографируют свои работы и отправляют их в чат. Также предусмотрен сценарий олимпиады по программированию с автоматической проверкой решений участников

\divider

\cvevent{Лендинг компании (завершено)}{ПО для оптимизации бизнеса}{Заказчик: ООО"ИНФОТИС"}{}
Одностраничный сайт, содержащий информацию о компании, продаваемые продукты. Также на сайте присутствует возможность оставить свои контакты для дальнейшей связи менеджера с клиентом

\divider

\cvevent{Чат-бот для анализа показаний датчиков (завершено)}{ПО для оптимизации бизнеса}{Заказчик: ООО"ИНФОТИС"}{}
Чат-бот, обеспечивающий обработку запросов пользователей к базе данных по заданным критериям и отправку
уведомлений администратору при сбоях устройства

\divider

\cvevent{Web платформа для анализа показателей датчиков (завершено) }{ПО для оптимизации бизнеса}{Заказчик: ООО"ИНФОТИС"}{}
Web-платформа позволяет анализировать показатели датчиков агро-зондов, экспортировать данные, конфигури-
ровать устройства, предоставлять доступ к данным для клиентов

\divider

\cvevent{Мобильное приложение для анализа показателей датчиков (завершено) }{ПО для оптимизации бизнеса}{Заказчик: ООО"ИНФОТИС"}{}
Мобильное приложение для android-смартфонов позволяет анализировать показатели датчиков агро-зондов, экспортировать данные, конфигурировать устройства, предоставлять доступ к данным для клиентов


% \divider


% \cvsection{Научные статьи}

\cvsection{Дипломы и письма благодарности}


\centering
\begin{tabular}{cc}
\qrcode[height=2in]{https://disk.yandex.ru/d/nHaDffMFVCbbbg} & \qrcode[height=2in]{https://drive.google.com/drive/folders/1ADxdwxoAEcJX6HLXBd7y3WoIzZzNxDIW?usp=drive_link}\\ \\
\huge{Яндекс диск} & \huge{Google drive}
\end{tabular}

\end{document}
